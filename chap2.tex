\chapter{LITERATURE REVIEW}

This chapter is composed of three main sections. The first section describes what a literature review is and explains its role in research. It highlights the importance of reviewing existing literature to understand the current state of knowledge on a topic and to build a foundation for further research. The second part of this chapter elaborates on the practical requirements to conduct a comprehensive literature review. It provides detailed steps on how to plan and execute the search for relevant literature. This includes choosing appropriate databases, using effective search strategies, and systematically analyzing and organizing the found information. Finally, the third section summarizes the key findings from the literature review and discusses their implications for the research field. It identifies gaps in current knowledge and suggests areas for future research. This section emphasizes how the literature review supports the research objectives and questions, providing a critical foundation for the subsequent chapters of the study.

\section{Introduction to Literature Review}
This section introduces the concept of a literature review and discusses its importance in research. A literature review serves as a critical summary of what the scientific community has published on a particular topic or subject. It provides an overview of current knowledge, allowing the researcher to position their own research within the existing literature.

\section{Conducting a Comprehensive Literature Review}
\subsection{Planning Your Literature Search}
Begin by defining clear objectives for your review. Determine the scope of your research and formulate research questions or hypotheses. Select databases and other sources relevant to your field of study. It is advisable to use a range of sources, including books, peer-reviewed journals, conference papers, and credible online resources.

\subsection{Searching for Relevant Literature}
Develop a search strategy using relevant keywords, phrases, and synonyms. Apply appropriate filters and Boolean operators (AND, OR, NOT) to refine your search results. Record your search strategies and results, which is essential for the reproducibility of your research.

\subsection{Screening and Selecting Literature}
Screen titles and abstracts for relevance to your research questions. Obtain and read full texts for selected sources. Use inclusion and exclusion criteria to systematically decide which articles to consider in your review.

\subsection{Analyzing and Synthesizing Information}
Organize the selected literature into thematic categories or according to methodological approaches. Analyze the findings and methodologies used in the literature to identify patterns, themes, and gaps in the research.

\subsection{Writing the Review}
Summarize the literature, linking it directly to your research questions and objectives. Discuss the significance of findings in relation to previous studies. Highlight any controversies, inconsistencies, and gaps in the literature. Present a critical analysis of the collected data, providing your interpretations and insights.




\begin{table}[h]
\centering
\caption{Comparison of Different Techniques}
\label{tab:technique_comparison}
\begin{tabular}{|c|m{6cm}|m{6cm}|}
\hline
\textbf{No.} & \textbf{Brief Note} & \textbf{Additional Details} \\ \hline
1 & Technique A: This technique is known for its speed and efficiency in processing large datasets. & Suitable for real-time applications. It can handle streaming data effectively but may require significant computational resources. \\ \hline
2 & Technique B: Often used for its high accuracy in classification tasks, especially in machine learning. & Requires extensive training data and is computationally expensive. It excels in scenarios where precision is critical. \\ \hline
3 & Technique C: Valued for its flexibility and adaptability across various domains. & While versatile, it may not provide the best performance in terms of speed or accuracy in specific applications. \\ \hline
\end{tabular}
\end{table}



\begin{table}[]
	\centering
	\caption{Reported results of the ternary sleepiness detection method in previous studies. }
	\label{tab:table3levelCompareOtherMinimal}
	\begin{tabular}{lllllll}
		\toprule[\heavyrulewidth] \toprule[\heavyrulewidth]
		No& Study   & Parameter(s) & Classifier   \\
		\midrule[\heavyrulewidth]
		1&\cite{Barua_2019} & 	EEG, EOG, CF& SVM \\
		2&\cite{Chai2019} & SWA & SVM \\
		3&\cite{Wang2016} & PERCLOS, SWA & 	MLO\\
		4&\cite{Zhao2015} & FE & NB \\
		5&\cite{Zhao2018} & FE & NB \\
		6&\cite{Zilin2017} & PERCLOS, EM & 	ANN\\
		7&\cite{Picot_2012} & 		EEG, EOG & 		CDR \\
		\hline	
	\end{tabular}
\end{table}

\section{Research Gap}

In the research gap section, it's important to find areas where questions still need answers or where current theories and methods don't fully solve the problems. This involves combining knowledge from previous studies you've reviewed, critically looking at their findings and methods, and spotting areas that haven't been fully explored yet. Doing this should help you see where you can make important improvements in the field, suggest new ways of researching, or address complex issues that haven't been dealt with thoroughly in past research.

\section{Chapter Summary}

A chapter summary in the literature review section is a brief overview of the key points covered in that chapter. It helps to consolidate the main findings, theories, and discussions presented. In this summary, you should clearly outline the most important ideas and how they relate to your research topic. This includes summarizing the debates, methodologies, and conclusions from various sources you've studied. The purpose of this summary is to give a clear and concise recap that helps readers understand the background and context of your study, and to set the stage for presenting the research gaps and questions your study aims to address.
