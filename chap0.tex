\raggedbottom 

\title{INOVATIVE: TO CREATE NEW MODULE FOR EFFECTIVENESS IN ADMINISTRATIVE STAFF WORKFLOW BY USING FULLY SISTEMIZE APPLICATION}

\TitleMalay{INOVATIF: MEWUJUDKAN MODUL BARU DALAM MENINGKATKAN KEBERKESANAN PROSES KERJA KAKITANGAN PENTADBIRAN DENGAN MENGGUNAKAN SEPENUHNYA APLIKASI BERSISTEMATIK}



\abstract{The acquisition of new vocabulary in a second language requires a series of repeated exposures. With emerging digital platforms and games such exposure can easily be provided. Due to the prevalence and ease of access of mobile technology, learners have demonstrated a high degree of dependence on digital tools such as smartphones for learning as compared to traditional approaches. This study is aimed at exploring whether a game-based thesaurus app could be used to improve the level of English language vocabulary among students in a public university in Malaysia. The findings reveal that students have less experience on using thesaurus apps compared to games or other language learning apps. Students prefer the use of mobile learning over traditional approach, prefer online platforms rather than mobile apps, and acquire or build up their vocabulary through watching movies and listening to music. Even though most students have adequate experience using mobile apps and games, they rarely use those platforms for learning purposes. This suggests that it is crucial to incorporate game elements into learning platforms particularly in learning English vocabulary to generate motivation and engagement to learners. Lecturers should therefore focus more on the explicit use of mobile digital technology in their teaching and learning classrooms. }



\abstractMalay{Pemerolehan perbendaharaan kata baharu dalam bahasa kedua memerlukan satu siri pendedahan berulang kali. Melalui platform digital dan permainan yang baru, pendedahan tersebut boleh disediakan dengan mudah. Oleh kerana kelaziman dan akses teknologi mudah alih adalah senang, pelajar telah menunjukkan tahap kebergantungan yang tinggi terhadap alat digital untuk belajar, seperti melalui telefon pintar berbanding dengan pendekatan tradisional. Kajian ini bertujuan untuk meneroka sama ada aplikasi tesaurus berasaskan gamificasi boleh digunakan untuk memperbaiki tahap penguasaan perbendaharaan kata dalam Bahasa Inggeris dalam kalangan pelajar universiti awam di Malaysia. Hasil kajian menunjukkan bahawa pelajar mempunyai pengalaman yang kurang dalam menggunakan aplikasi tesaurus berbanding pembelajaran bahasa berasaskan permainan atau aplikasi lain. Pelajar lebih suka menggunakan pembelajaran mudah alih berbanding pendekatan tradisional, lebih suka platform dalam talian dan bukannya aplikasi mudah alih, dan memperoleh atau membina perbendaharaan kata mereka melalui menonton filem dan mendengar muzik. Meskipun fakta menyatakan bahawa kebanyakan pelajar mempunyai pengalaman yang mencukupi dalam penggunaan aplikasi mudah alih dan permainan, namun mereka jarang menggunakan platform tersebut untuk tujuan pembelajaran. Ini menunjukkan bahawa adalah penting untuk menggabungkan elemen permainan ke dalam platform pembelajaran khususnya dalam pembelajaran perbendaharaan kata Bahasa Inggeris sebagai satu cara untuk menjana motivasi dan penglibatan pelajar. Oleh yang demikian, pensyarah perlu memberi tumpuan lebih kepada penggunaan yang jelas terhadap teknologi digital mudah alih dalam pengajaran dan pembelajaran di bilik darjah. }


\author{RODNEY PETRUS BALANDONG}
% \programfacul{INDUSTRIAL PHYSICS}
% \faculty{FACULTY OF SCIENCE AND NATURAL RESOURCES}


\degree{ BACHELOR OF SCIENCE WITH HONORS} % or \degree{Master of Science} 
\mainsupervisor{AP. Dr. Zoro}



\matricno{2023456}


\copyrightyear{\number\the\year} % or \copyrightyear{20xx}


\declarationLLM{I acknowledge that during the preparation of this thesis, I have utilized Large Language Models (LLMs) as a supplementary tool to assist in various aspects of my research and writing process. Specifically, I have used LLMs for paraphrasing and refining my writing to improve clarity, coherence, and readability while ensuring the original meaning remains intact. Additionally, I have leveraged LLMs to generate general ideas and structure my thoughts more effectively, using them as a brainstorming aid rather than as a replacement for original critical thinking. LLMs have also been helpful in summarizing complex topics, identifying key points from lengthy articles, and suggesting alternative ways to present information. Furthermore, I have used LLMs to check grammar and syntax, ensuring that my writing adheres to academic standards. In some instances, LLMs provided guidance on citation formats and referencing best practices, although all sources have been manually verified to maintain academic integrity. Importantly, I have critically evaluated all AI-generated content and ensured that it aligns with my own understanding and research findings.}


\penghargaan{It is good to tha}

%%%%% IMPORTANT. DO NOT PLACE WHAT HAVE BEEN DECLARED ABOVE, AFTER THIS CELL
\beforepreface %dont change this

\afterpreface  %don't change this